\section{Main Functionalities}
%What are the main functionalities of the web app? what services does it offer and how it is organized?
\hspace{1cm} The pharmacy website offers several key functionalities for its staff members to manage business operations. The admin creates accounts for staff members, who can then access the system using their login credentials to perform various tasks such as browsing products, placing orders with suppliers, and managing prescriptions. 
Staff members can also upload prescriptions on behalf of customers, and pharmacists can view the prescription and dispense the medication accordingly.\par
Another important feature is storage management. The website provides a system for managing the
storage of drugs and materials. It tracks the quantity available and creates alerts for low stock levels.
This helps ensure that there is always sufficient stock on hand.\par
Order management is another critical functionality offered by the website. Staff members can place orders with suppliers for drugs and materials through the website, and once the orders are received
from the suppliers, pharmacists can review and fulfill the orders by dispensing the requested items from the storage. This process is streamlined and efficient, ensuring that customers receive their orders in a timely manner.\par
Supplier management is also a key component of the website. Suppliers who provide drugs and materials to the pharmacy are managed through the system. The website stores information about the supplier and the products they supply. Reports can be generated to track supplier performance and assess the quality of the products received.\par
Reporting is another vital functionality provided by the pharmacy website. The system generates reports on various aspects of the business such as sales and storage levels. These reports help the pharmacy to make informed decisions about stocking and pricing products, as well as identifying trends in customer behavior.\par
Payment processing is an essential part of the website. Staff members can record how the customer intends to pay for their order, such as cash or credit card, but the actual payment is not processed through the website.\par
Finally, receipt management is an important feature. The system generates receipts for each order and stores them for future reference. This helps ensure that the pharmacy maintains accurate records and can easily track customer transactions.\par
Role-based access control is also in place to ensure that only authorized personnel can perform
certain functions within the system. This enhances security and ensures that sensitive data is protected.
\
\\
\
\\
\
\\
\\
\