\section{Introduction}
\label{sec:introduction}

Information retrieval (IR) is a rapidly evolving research area concerned with the organization, storage, and retrieval of information from large-scale document collections. The primary goal of IR systems is to assist users in finding relevant information that satisfies their information needs, despite the growing volume and complexity of data available on the internet \cite{ZhangSanner2021}. This paper presents our approach to developing a search engine for processing documents written in English and French languages, using the Lucene library \cite{lucene3.5.0} and its methods to provide a custom search solution. Apache Lucene is an open-source software library developed by the Apache Software Foundation, primarily used for implementing search engines that work with text-based inputs. The library provides a streamlined API that facilitates the indexing and searching of large collections of text documents \cite{lucene2010apache}.

Recent years have witnessed the emergence of several prominent IR systems, such as Hello-IR \cite{chimetto2022seupd} and Hello-Tipster \cite{BaruscoEtAl2022}, which have made significant contributions to the field. In our project, we have carefully analyzed these previous works to identify their strengths and limitations. Building upon this analysis, we have devised our own search engine that addresses some of the identified challenges and offers improved performance across various metrics.

Our methodology includes a comprehensive Experimental Setup, which makes use of the Qwant search engine's collection of queries and documents [4]. This collection serves as the official test collection for the 2023 LongEval Information Retrieval Lab (https://clef-longeval.github.io/) \cite{CLEFLongEval}, and consists of test datasets for two sub-tasks: short-term persistence (sub-task A) and long-term persistence (sub-task B). Our experiments are designed to evaluate the performance of our search engine in these sub-tasks, and the results are discussed in detail in the Results and Discussion section.

In the Conclusions and Future Work section, we summarize our findings and outline the potential areas for future research in the field of information retrieval. By developing a custom search engine that leverages the capabilities of the Lucene library, our work contributes to ongoing advancements in the field of information retrieval and offers new insights for the processing of multilingual document collections.


The paper is organized as follows: Section~\ref{sec:methodology} describes our approach; Section~\ref{sec:setup} explains our experimental setup; Section~\ref{sec:results} discusses our main findings; finally, Section~\ref{sec:conclusion} draws some conclusions and outlooks for future work.